\section{\hei{暑假第三周工作报告}}

\subsection{\hei{本周工作摘要}}
\begin{itemize}
\item{本周前两天看了一些llvm的文档,主要是关于clang的一些简介和使用文档,clang是面向C/C++/ObjC ,作为llvm的前端,想比较gcc,有更快的编译速度、较少的内存使用,对于编译报错有更好的形式。}
\item{之前对于c++,不是太了解,参考《c++ Primer》看了一部分c++的知识,大概了解了类,namespace的知识,类感觉有点像c中的结构体,类有定义了相关的操作。namespace,是指标识符的各种可见范围。C++标准程序库中的所有标识符都被定义于一个名为std的namespace中,使用标志符有3种选择,1、直接用std::cout...,2、使用using关键字,using std::cout;cout....,3、using namespace std; cout....关于其他知识还要继续看。}
\item{参考《程序员的自我修养—链接、装载与库》,还看了一点关于静态链接和动态链接的知识,静态链接:源程序经过编译生成目标文件,根据目标文件各个段的长度,属性和位置,将他们合并,并计算出输出文件中各个段的长度与位置,并建立映射关系,符号统一放到一个全局符号表中。之后读取输入文件中的段的数据、重定位信息,并且进行符号解析与重定位、调整代码中的地等。动态链接相比较而言,兼容性和扩展性更好,可以共享同一库函数。对于动态链接的过程整体上还不是很清楚,还在看。}
\item{了解了graphviz画图工具,关于dot语言画图的文档已经写了一部分,在节点上点击出现相关信息这个问题还没有解决。}
\subsection{\hei{问题与不足}}
关于graphviz画图还需要继续深入的了解。
\end{itemize}
\subsection{\hei{下周计划}}
1)继续寻找上述问题的解决方案。


2)看相关的代码。


3)抽时间继续看C++和动态链接。

