\section{\hei{暑假第一周工作报告(2013.7.15-2013.7.21)}}

\subsection{\hei{工作摘要}}

\begin{itemize}
\item{由于以前对于linux系统接触的比较少,一开始对于许多基本操作感觉很陌生,所以本周一开始对于linux系统基本操作和一些常用的软件做了一些了解,经过几天的适应,逐渐掌握了linux环境下的基本操作。}

\item{本周主要还是阅读了关于dlinux的几篇论文,主要阅读了《A Virtual Memory Foundation for Scalable Deterministic Parallelism》和《Lazy Tree Mapping: Generalizing and Scaling Deterministic Parallelism》,对于《Efficient System-Enforced deterministic Parallelism》粗略的浏览了一遍,结合曹慧芳学姐的讲解,现在对于spmc memory model和lazy tree mapping有了整体的认识.这一周由于没有看代码,对于一些细节问题还有待于继续深入的了解。}

\item{一开始阅读论文时,对于其中的许多专业术语很陌生,例如MPI,Pthreads等等,期间对于这些术语我查阅了许多资料,主要看了MPI,。通过刘道琛的介绍,我阅读了\href{https://computing.llnl.gov/tutorials/mpi/}{Message Passing Interface (MPI)},并结合课件\href{https://http://wenku.baidu.com/view/ee8bf3390912a216147929f3.html}{MPI编程简介},对于MPI有了初步的了解,通过几个实例,重点了解了几个基本的MPI函数,以及点对点的通信(p to p).另外对于并行计算也进行了一些了解,主要阅读了\href{https://computing.llnl.gov/tutorials/parallel_comp/}{Introduction to  Parallel Computing 
}。}
\end{itemize}

\subsection{\hei{存在的问题与不足}}

在一周的学习中,遇到这样或那样的难题,通过其他同学的帮助,得到了解决,在中间学到了很多的东西,但是总的看来,还是由于基础知识薄弱引起的,以后要多阅读一些资料,弥补自己的不足。现在还是有点迷茫,自己应该看什么样的书。

\subsection{\hei{下一周计划}}
1)看dlinux源代码。

2)抽时间看linux内核设计与实现。


