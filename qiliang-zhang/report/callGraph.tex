\section{\hei{函数调用图}}



================================================================



\subsection{\hei{第一部分:收集所需要的信息}}
\kai{信息的存放}
\begin{itemize}

\item{typedef DenseMap<Module*, SetVector<Function*> > FunctionList;维护Module<->Function关系,存放这函数信息。}

\item{typedef DenseMap<Function*, SetVector<CallInst*> > CallInstList;维护Function<->CallSite关系,存放着函数调用的调用函数信息。}

\item{typedef std::map<Function*, std::map<std::string, int> > CallInstRef;维护CallInst的引用计数信息,在某一个函数中,调用函数被引用的次数,用于画图时,添加边的信息。}

\item{typedef DenseMap<Function*, std::pair<bool, bool> > FunctionStack;维护函数的状态信息,是否被访问过。} 

\end{itemize}
\kai{信息收集的接口}

\begin{itemize}

\item{insertFunctionToList(Module *M, Function *F):将模块M中的F插入到对应的列表中。}
\item{insertCallInstToList(Function *F, CallInst *CI),将模块F中的调用函数插入到对应的列表中。}
\item{insertCallInstRef(Function *F, std::string CallName, int count),收集F调用CallName的次数。}
\item{getCallInstofFunction(Function *F):由给出的F,获得他调用的函数集合。}
\item{getCountofCallInst(Function *F, std::string callName):通过F先得到std::map<std::string, int>,在利用callName获取函数F调用callName的次数。}
\item{insertFunctionToStack:保存着函数F的访问状态。}
\item{isFunctionVisiting(Function *F)和isFunctionVisited(Function *F),保存在FunctionStack.second,即std::pair<bool, bool>中。}
\end{itemize}


\kai{收集信息的过程}
\begin{itemize}
\item{在main.cpp中循环访问的ModuleList,对每一个Module进行访问,通过collectCallSiteInfo(Module *M)收集模块中的函数信息。}

\item{collectCallSiteInfo(Module *M):对于某一个模块M,首先遍历模块中顶层函数,并插入到Flist,当遇到一个函数时收集其中的CallInst,保存在F对应的CIList中。}

\end{itemize}
\par{由上面两步,可以得到FunctionList和CallInstList,之后在写.dot文件中在进行相应的处理。}

\par{其他接口}
\par{1.getFunctionName(Function *F),由于函数在内部的命名规则,\_Z + 函数名的长度 + 函数名 + 后面表示参数的字符串,这个接口返回的是函数名。}
\par{2.getCallInstName(CallInst *I),返回被调用函数的函数名称。}

\par{3.getCallRef(CallInstList *CIL)函数用于消除DenseMap中的重复项目;对于没一个函数F,迭代访问F对应的callinst的每一项,得到调用指令被引用的次数。调用insertCallInstRef(F, getCallInstName(A), CallInstCount)保存被调用函数和被调用的次数的信息。}


================================================================================


\subsection{第二部分:dot图的生成}
\par{1.从给出的名字出发调用generateDotGraph(std::string RootFunctionName),写相应的.dot文件}
\par{2.生成dot图,主要是利用前面收集到的\en{CallInstList* CIL=BIDB.getCallInstList()},利用BIDB.getCallInstRef()将CallInstList,中的重复项去除,并添加调用的次数信息。}
\par{3.经过步骤2后,所得到一个函数链以及函数调用相应的函数链。之后,由给出的根节点rootFunctionName,在CIL中查找对应的rootFunction,然后写dot文件的头部(参考怎么画dot图)。}
\par{4.从rootFunction出发,写相关的点和边的信息,对于写相关的点和边的信息,调用writeNode(File, rootFunction, CIL)(关于怎么写点和边的信息参考dot图的画法),接着访问SetVector<CallInst*>中rootFunction调用的每一个函数,递归调用writeNode,到达底层函数(即没有调用其他函数的函数)或已经被访问过的函数返回。在递归调用的同时,设置函数的访问信息,返回时,设置此函数的状态已经被访问。若正在被访问的函数,若再一次被访问,则表示有循环调用了,之后写出rootFunction到每一个调用函数的的边信息,边的信息通过getCountofCallInst(F, getCallInstName(CI))获得。访问完后,并设置函数的状态信息。}
\par{5.接着写dot文件的尾部。}
\par{注:关于写.dot文件的格式,可参考llvm-work/doc/intro/intro/Graphwriter.tex }
