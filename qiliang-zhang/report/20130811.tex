\section{\hei{暑假第四周工作报告}}

\subsection{\hei{本周工作概要}}

\begin{itemize}

\item{关于dot画图,文档放在了llvm-work/doc/intro/目录下。}

\item{主要熟悉了llvm IR的表示方法,参考这源代码,了解llvm IR每条命令所代表的含义以及与源代码的对应关系,自己也写了一些for、while、do while循环语句和if else、switch条件判断语句的例子,明白各种语句对应的模块(例如for循环,对应for.cond、for.body、for.inc、for.end),但是有些细节还是不明白,不明白使用\%retval这个变量的条件,还有各个基本块后面的数字表示等。用llc -march=cpp -o xx.cpp xx.ll命令,可以由xx.ll文件得到描述建立模块和基本块的xx.pp文件,关于打印相关信息的问题,现在还没有头绪。}

\item{看了llvm/Support/GRaphWriter.h,向.dot文件写入相关的信息,类GraphWriter提供了writeGraph(),writeHeader,writeNodes(),writeFooter(),writeNode(),writeEdge(),emitEdge()

writeGraph():调用writeHeader(Title)将开头写入.dot文件中,之后调用writeNodes()将节点写入,再调用addcustomGRaphFeatures为图添加附加的信息,writeFooter()写文件的结尾。

void writeHeader(const std::string \&Title):写文件的开头digraph Title/Graphname/unnamed \{ rank=BT;label="Title/Graphname"。

writeFooter():写结尾"\}"。

void writeNodes():通过迭代器,循环扫描所有节点,调用writeNode(*I)写入节点信息。

void writeNode(NodeType *Node):为节点*node添加shape、label、地址信息、节点描述信息、以这个点为起点的边si、以这个点为终点的边di,通过迭代,调用witeEdge画这个点与其他点的边。

void writeEdge(NodeType *Node, unsigned edgeidx, child\_iterator EI):得到DestPort,TargetNode。调用emitEdge输出边的信息,写入Node SrcNodeID :s SrcNodePort-> Node DestNodeID :d DestNodeID[ Attrs ];

另外还提供了输出简单节点的接口emitSimpleNode。
 
}
\item{看了c++ Primer plus,简单过了一遍前面的基础知识,看到了地10章对象与类。}
\end{itemize}

\subsection{\hei{问题与不足}}
代码的表示不太习惯,看的有点晕。

\subsection{\hei{下一周的计划}}
1)继续调研llvm IR与源代码的关系。

2)继续看画图的相关的类与代码。

3)接着看c++。
